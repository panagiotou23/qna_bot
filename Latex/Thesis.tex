\documentclass{report}
\usepackage[utf8]{inputenc}
\usepackage{graphicx}
\usepackage{amsmath}
\usepackage{amsfonts}
\usepackage{amssymb}
\usepackage{hyperref}
\usepackage{lipsum}
\usepackage{geometry}
\usepackage{natbib}
\usepackage{polyglossia}
\usepackage{csquotes}
\usepackage{fontspec}

\setdefaultlanguage{greek}
\setotherlanguage{english}

\newfontfamily\greekfont[Script=Greek]{GFS Artemisia}

\title{Διπλωματική}
\author{Αλέξανδρος Παναγιώτου}
\begin{document}

\begin{titlepage}
    \centering
    \vspace*{1cm}
    {\LARGE Αλέξανδρος Παναγιώτου \par}
    \vspace*{0.5cm}
    {\Large Αριστοτέλειο Πανεπηστήμιο Θεσσαλονίκης \par}
    \vspace{2cm}
    {\Huge\bfseries Διπλωματική Εργασία \par}
    \vspace{2cm}
    {\Large Easy to Use Java QnA Bot Library \par}
    \vspace*{1cm}
    {\large Ηλεκτρολόγων Μηχανικών και Μηχανικών Υπολογιστών \par}
    \vspace{2cm}
\end{titlepage}

\tableofcontents 

\chapter*{Πρόλογος}
\addcontentsline{toc}{chapter}{Πρόλογος}

Στο σημερινό ταχέως εξελισσόμενο ψηφιακό τοπίο, η ανάπτυξη ευφυών ρομπότ συνομιλίας και ερωταποκρίσεων (QnA) διαδραματίζει καθοριστικό ρόλο στη βελτίωση των εμπειριών των χρηστών και στη διάδοση της γνώσης \cite{eliza, parry}. Η παρούσα διπλωματική παρουσιάζει μια ολοκληρωμένη διερεύνηση της ανάπτυξης μιας φιλικής προς τον χρήστη βιβλιοθήκης Chat Bot  και QnA Bot που βασίζεται σε Java, με έμφαση στην επίδραση των διαφόρων δομικών στοιχείων στην ακρίβεια του μοντέλου. Στην συνέχεια συγκρίνει τα state of the art μοντέλα αξιολόγησης των Large Language Models (LLMs) με τα οποία θα συγκρίνει μετέπειτα την επίδραση των δομικών στοιχείων στην επίδοση του συστήματος.

Στόχος του έργου είναι η αξιολόγηση της αποτελεσματικότητας διαφορετικών συνδυασμών δομικών στοιχείων στην ανάπτυξη chat και QnA bots. Μέσω αυτής της ανάλυσης, στοχεύουμε να παράσχουμε πολύτιμες πληροφορίες σχετικά με την αρχιτεκτονική και τις σχεδιαστικές επιλογές που μπορούν να βελτιώσουν σημαντικά την απόδοση αυτών των συστημάτων συνομιλίας AI. Η μελέτη μας περιλαμβάνει μια σειρά από τεχνικές μοντελοποίησης\cite{ada, bert, gpt, roberta} και μετρικές αξιολόγησης για να μετρήσουμε την ποιότητα των bots που αναπτύχθηκαν \cite{squad} 

Η μεθοδολογία που χρησιμοποιείται στην παρούσα έρευνα περιλαμβάνει ιστορική αναδρομή, διερεύνηση της κατάστασης της τεχνολογίας στον τομέα των chat και QnA bots, και θεωρητική ανάλυση των υποκείμενων αρχών \cite{book}. Επιπλέον, εμβαθύνουμε σε λεπτομερείς επεξηγήσεις των διαφόρων μοντέλων και δομικών στοιχείων που χρησιμοποιούνται στη βιβλιοθήκη, παρέχοντας μια σταθερή βάση για την κατανόηση των πειραμάτων και των αποτελεσμάτων μας \cite{xiaoice, duplex}.

Το αποτέλεσμα της εργασίας αυτής είναι μια ολοκληρωμένη σύγκριση διαφορετικών συνδυασμών δομικών στοιχείων, με έμφαση στον αντίκτυπό τους στην ακρίβεια. Τα ευρήματα αυτής της μελέτης έχουν σημαντικές συνέπειες στην γρήγορη και αποτελεσματική ανάπτυξη ευφυών ρομπότ, συστημάτων συνομιλίας και πλατφορμών QnA, συμβάλλοντας στην πρόοδο της συνομιλιακής τεχνητής νοημοσύνης.

\chapter{Εισαγωγή}

Σε μια εποχή που χαρακτηρίζεται από τον πολλαπλασιασμό της ψηφιακής επικοινωνίας και της ανταλλαγής πληροφοριών, η ανάπτυξη της συνομιλιακής τεχνητής νοημοσύνης έχει αποκτήσει εξέχουσα σημασία. Τα chatbots και γενικότερα οι εφαρμογές που απαντούν σε ερωτήσεις βρίσκονται στην πρώτη γραμμή της επανάστασης στον τρόπο με τον οποίο τα άτομα αλληλεπιδρούν με πληροφορίες, υπηρεσίες και συστήματα \cite{greek_ref}. Αυτοί οι ευφυείς πράκτορες είναι ικανοί να κατανοούν και να απαντούν σε ερωτήματα φυσικής γλώσσας, διευκολύνοντας έτσι την απρόσκοπτη αλληλεπίδραση και την πρόσβαση στη γνώση. Αυτή η εισαγωγή παρέχει μια επισκόπηση του πλαισίου, της σημασίας και των στόχων της έρευνας, θέτοντας τις βάσεις για μια διεξοδική έρευνα σχετικά με την ανάπτυξη μιας βιβλιοθήκης bot chat και QnA που βασίζεται σε Java και είναι φιλική προς τον χρήστη.

\section{Στόχοι και πεδίο εφαρμογής της έρευνας}

Αυτή η έρευνα έχει ως στόχο να διερευνήσει την επίδραση διαφορετικών συνδυασμών δομικών στοιχείων στην ακρίβεια των bots συνομιλίας και QnA. Τα δομικά αυτά στοιχεία είναι τα εξής:

\begin{itemize}
  \item Ο τρόπος με τον οποίο σπάει ένα μεγάλο κείμενο προκείμένου να μπορέσει να κωδικοποιηθεί (Sentence/Word Based)
  \item Ο αλγόριθμος ενσωμάτωσης του κειμένου αυτού (ADA/BERT)
  \item Ο αλγόριθμος εύρεσης το K ποιό κοντινών γειτόνων (Απόσταση Συνημητόνου/Ευκλείδια)
  \item Η επιρροή του Κ
  \item Το μοντέλο ολοκλήρωσης που θα απαντήσει στο τελικό χρήστη (ChatGPT-3-5, RoBERTa)
\end{itemize}


Στην υπόλοιπη εργασία θα αναπτυχθούν τα εξής:

\begin{itemize}
  \item Διερεύνηση της ιστορικής εξέλιξης των chatbots και των συστημάτων απάντησης ερωτήσεων
  \item Ανάλυση της τρέχουσας κατάστασης της τέχνης στη συνομιλιακή τεχνητή νοημοσύνη 
  \item Παροχή θεωρητικής βάσης για την κατανόηση των μοντέλων και των δομικών στοιχείων που χρησιμοποιούνται στη βιβλιοθήκη
  \item Ανάπτυξη μιας φιλικής προς τον χρήστη βιβλιοθήκης chat και QnA bot βασισμένης σε Java .
  \item Αξιολόγηση της απόδοσης της βιβλιοθήκης συγκρίνοντας διαφορετικούς συνδυασμούς δομικών στοιχείων
\end{itemize}

\begin{thebibliography}{10}
\bibitem{book} Tunstall et al. "Natural Language Processing with Transformers"
\bibitem{eliza} Weizenbaum, Joseph. "ELIZA – A Computer Program for the Study of Natural Language Communication between Man and Machine." Communications of the ACM, 1966.
\bibitem{parry} Colby, Kenneth M. "Simulation of Behaviour of Psychopathological Patients." ACM Computing Surveys, 1973.
\bibitem{alice} Wallace, Richard S. "The Anatomy of ALICE." Minds and Machines, 1999.
\bibitem{xiaoice} Shen, Zho, et al. "A Knowledge-Grounded Neural Conversation Model." AAAI Conference on Artificial Intelligence, 2017.
\bibitem{duplex} Minkov, Einat, et al. "Language Models are Few-Shot Learners." Advances in Neural Information Processing Systems, 2020.
\bibitem{squad} Rajpurkar, Pranav, et al. "SQuAD: 100,000+ Questions for Machine Comprehension of Text." Empirical Methods in Natural Language Processing, 2016.
\bibitem{bert} Devlin, Jacob, et al. "BERT: Bidirectional Encoder Representations from Transformers." arXiv, 2018.
\bibitem{gpt} Radford, Alec, et al. "Improving Language Understanding by Generative Pre-training." OpenAI, 2018.
\bibitem{t5} "Exploring the Limits of Transfer Learning with a Unified Text-to-Text Transformer." Raffel, Colin, et al. 
\bibitem{roberta} "RoBERTa: A Robustly Optimized BERT Pretraining Approach", Liu, Myle et al. 
\bibitem{ada} "GPT3-to-plan: Extracting plans from text using GPT-3", Olmo et al. 
\bibitem{greek_ref} "An Overview of Chatbot Technology", Eleni Adamopoulou \& Lefteris Moussiades 

\end{thebibliography}

\end{document}
